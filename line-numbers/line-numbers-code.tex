% -----------------------------------------------------------
% demonstrates how to attach line numbers to code snippets, 
% and build according references
%
% (C) 2015 Frank Hofmann, Berlin, Germany
% Released under GNU Public License (GPL)
% email frank.hofmann@efho.de
% -----------------------------------------------------------

% set preamble
% - define document class: article
\documentclass{article}

% - do not change paper size
% - do not change character encoding: UTF-8 (Unicode)
% - do not change document grammar checking: english

% - load additional package lineno
\usepackage{lineno}

% document content
% - start document here
\begin{document}

% paragraph without line numbers
\noindent The makefile \texttt{Makefile} contains the following lines of code: \\


% paragraph two with line numbers
\linenumbers
\noindent\texttt{%
\linelabel{all}all: html pdf \\
~\\
\linelabel{html}html einsteiger.html: *.asciidoc Makefile \\
asciidoc -a toc -a toclevels=3 einsteiger.asciidoc \\
~\\
\linelabel{pdf}pdf einsteiger.pdf: *.asciidoc Makefile \\
a2x -L -f pdf einsteiger.asciidoc
}

% disable line numbers
\nolinenumbers
~\\
Calling \texttt{make all} results in the creation of both an HTML, and a
PDF file (see line~\ref{all}). The interpreter jumps to the according
labels as defined in lines~\ref{html}, and ~\ref{pdf}.

% - end document here
\end{document}
