
% -----------------------------------------------------------
% demonstrates how to change between various font shapes
%
% (C) 2015 Frank Hofmann, Berlin, Germany
% Released under GNU Public License (GPL)
% email frank.hofmann@efho.de
%
% Paragraph content taken from Wikipedia:
% https://en.wikipedia.org/wiki/GNUstep
% -----------------------------------------------------------

% set preamble
% - define document class: article
\documentclass{article}

% - do not change paper size
% - do not change character encoding: UTF-8 (Unicode)
% - do not change document grammar checking: english

% document content
% - start document here
\begin{document}

\section{Emphasise text}
\subsection{Bold face, and italic}
\textbf{GNUstep} is a free software implementation of the \textit{Cocoa} (formerly
\textit{OpenStep}) Objective-C frameworks, widget toolkit, and application
development tools for Unix-like operating systems and Microsoft Windows.
It is part of the GNU Project.

\subsection{Slanted, and small capitals}
\textsl{GNUstep} is a free software implementation of the \textsc{Cocoa} (formerly
\textsc{OpenStep}) Objective-C frameworks, widget toolkit, and application
development tools for Unix-like operating systems and Microsoft Windows.
It is part of the \textsc{GNU Project}.

\section{Font style}
\subsection{Sans serif}
\textsf{GNUstep is a free software implementation of the Cocoa (formerly
OpenStep) Objective-C frameworks, widget toolkit, and application
development tools for Unix-like operating systems and Microsoft Windows.
It is part of the GNU Project.}

\subsection{Serif}
\textrm{GNUstep is a free software implementation of the Cocoa (formerly
OpenStep) Objective-C frameworks, widget toolkit, and application
development tools for Unix-like operating systems and Microsoft Windows.
It is part of the GNU Project.}

\subsection{Typewriter}
\texttt{GNUstep is a free software implementation of the Cocoa (formerly
OpenStep) Objective-C frameworks, widget toolkit, and application
development tools for Unix-like operating systems and Microsoft Windows.
It is part of the GNU Project.}

% - end document here
\end{document}
