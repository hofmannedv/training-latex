
% -----------------------------------------------------------
% demonstrates how to write a book class LaTeX document
%
% (C) 2015 Frank Hofmann, Berlin, Germany
% Released under GNU Public License (GPL)
% email frank.hofmann@efho.de
%
% Paragraph content taken from Wikipedia:
% https://en.wikipedia.org/wiki/Skye
% -----------------------------------------------------------

% set preamble
% - define document class: book
\documentclass{book}

% - do not change paper size
% - do not change character encoding: UTF-8 (Unicode)
% - do not change document grammar checking: english

% document content
% - start document here
\begin{document}

% division of content: numbered headline for a chapter
\chapter{The Isle of Skye}

% section one (numbered)
\section{Introduction}
Skye or the Isle of Skye is the largest and most northerly major island
in the Inner Hebrides of Scotland. The island's peninsulas radiate from
a mountainous centre dominated by the Cuillins, the rocky slopes of
which provide some of the most dramatic mountain scenery in the country.
Although it has been suggested that the Gaelic Sgitheanach describes a
winged shape there is no definitive agreement as to the name's origins.

% section two (numbered)
\section{Occupation of the island}
The island has been occupied since the Mesolithic period and its history
includes a time of Norse rule and a long period of domination by Clan
MacLeod and Clan Donald. The 18th-century Jacobite risings led to the
breaking up of the clan system and subsequent Clearances that replaced
entire communities with sheep farms, some of which also involved forced
emigrations to distant lands. Resident numbers declined from over 20,000
in the early 19th century to just under 9,000 by the closing decade of
the 20th century. Skye's population increased by 4 per cent between 1991
and 2001. About a third of the residents were Gaelic speakers in 2001,
and although their numbers are in decline this aspect of island culture
remains important.

% - end document here
\end{document}
